\documentclass{beamer}

\usetheme{Antibes}

%\usetheme{split}
%\usetheme{Madrid}
%\usetheme{Berkeley}

\usecolortheme[RGB={120,0,0}]{structure}
\setbeamertemplate{blocks}[rounded][shadow=true]
\setbeamertemplate{footline}[frame number]
\usepackage[utf8]{inputenc}   % pacote para acentuao
\usepackage{graphicx}
\usepackage{subcaption}

%\usepackage[latin1]{inputenc}

\beamertemplateballitem
\beamertemplatenavigationsymbolsempty

\begin{document}

\title{Assessing the Computation and Communication Overhead of Linux Containers for HPC Applications}
\author{
\large 
\underline{Guilherme R. Alles}, Lucas M. Schnorr, Alexandre Carissimi\\
\small
\vspace*{0.5cm}
Instituto de Informática\\ Universidade Federal do Rio Grande do Sul \\\medskip
\includegraphics[width=.15\textwidth]{cnpq.png}%
\hspace{.5 cm}%
\includegraphics[width=.15\textwidth]{gppd-logo.png} \\\vspace{-.5cm}
}
\date{Simpósio de Sistemas Computacionais de Alto Desempenho\\
  2 de outubro de 2018}

\frame{\titlepage}

\section{Introduction and Objectives}
\frame{\frametitle{Introduction}
    \begin{itemize}
        \item HPC clusters are \textbf{highly heterogeneous}
        \begin{itemize}
            \item Hardware configurations, software stacks, usage policies...
        \end{itemize}
        \pause
        \vspace*{0.3cm}
        \item Designing experiments for portability requires investing time
        \begin{itemize}
            \item Operating system specifics
            \item Dependency management
        \end{itemize}
        \pause
        \vspace*{0.3cm}
        \item What if someone wants to reproduce your experiments?
    \end{itemize}
}

\frame{\frametitle{Introduction}
    Possible solution: virtual machines!\pause But...
    \begin{itemize}
        \item Requires a \textbf{hypervisor}. Which one?
        \begin{itemize}
            \item Type 1, Type 2, paravirtualization
            \item What if different clusters support different hypervisors?
        \end{itemize}
        \item Multi-gigabyte system images
        \item \textbf{Overhead}
    \end{itemize}
}

\frame{\frametitle{Introduction}
    Why can't we use Linux Containers?
    \begin{itemize}
        \pause
        \vspace*{0.2cm}
        \item No hypervisor overhead
        \pause
        \vspace*{0.2cm}
        \item Images take much less space
        \pause
        \vspace*{0.2cm}
        \item APIs integrated in the Linux Kernel
        \pause
        \vspace*{0.2cm}
        \item Flexibility provided can yield performance gains at a small development cost
    \end{itemize}
}

\frame{\frametitle{Objectives}
    \begin{itemize}
        \item Measure \textbf{viability} of applying container technology to HPC
        \pause
        \begin{itemize}
            \item Computation overhead
            \item Communication overhead
        \end{itemize}
        \pause
        \vspace*{0.5cm}
        \item Contrast two popular container environments and their \textbf{workflows}
        \begin{itemize}
            \item Docker
            \item Singularity
        \end{itemize}
    \end{itemize}
}

\frame{\frametitle{Outline}\tableofcontents[hideallsubsections]}

\section{Linux Containers}
\frame{\frametitle{Linux Containers}
    \begin{itemize}
        \item Operating System level virtualization
        \pause
        \item \textbf{Resource control} and \textbf{isolation} build into the Linux Kernel
        \begin{itemize}
            \item \textbf{Cgroups} API
            \item \textbf{Namespaces} API
        \end{itemize}
    \end{itemize}
    \vspace*{0.2cm}
    \centering
    \includegraphics[width=.3\textwidth]{conteineres.png}\\
    \footnotesize(translate this figure, probably enhance it)
}

\frame{\frametitle{Docker}
    \vspace*{-0.7cm}
    \begin{figure}
        \centering
        \includegraphics[width=.3\textwidth]{docker-logo.png}
    \end{figure}
    \vspace*{-0.2cm}
    \begin{itemize}
        \item Widely used in the industry
        \begin{itemize}
            \item Microservices virtualization
        \end{itemize}
        \item Standard in cloud infrastructure providers such as AWS, GCP and Azure
        \pause
        \item Enforces the virtualization of every namespace provided by the Linux Kernel
        \begin{itemize}
            \item \textbf{User}, file system, \textbf{network}, IPC, PID
            \item Implications on the use case (elaborate on that)
        \end{itemize}
    \end{itemize}
}

\frame{\frametitle{Singularity}
    \vspace*{-0.5cm}
    \begin{figure}
        \centering
        \includegraphics[width=.1\textwidth]{singularity-logo.png}
    \end{figure}
    \vspace*{-0.2cm}
    \begin{itemize}
        \item Linux Containers for HPC
        \vspace*{0.2cm}
        \item Alternative to Docker drawbacks for shared environments
        \pause
        \vspace*{0.2cm}
        \item Objective: \textbf{secure} container environment for HPC clusters
        \pause
        \vspace*{0.2cm}
        \item Virtualizes \textbf{only the necessary namespaces}. Everything else is kept optional.
        \begin{itemize}
            \item \textbf{file system}
        \end{itemize}
    \end{itemize}
}

\section{Experimental Design}
\frame{\tableofcontents[currentsection,hideothersubsections]}

\frame{\frametitle{Testbed}
    \begin{itemize}
        \item Grid5000's Graphene clusters
        \begin{itemize}
            \item Intel Xeon X3340, 4 cores @ 2.53GHz
            \item 16GB DDR3 RAM
            \item Gigabit Ethernet interconnect
        \end{itemize}
        \pause
        \vspace*{0.5cm}
        \item Debian 9 (\textit{stretch}) Linux
        \begin{itemize}
            \item Containers running on top of the native environment
            \item Docker containers connected through Docker Swarm \textit{overlay}
            \begin{itemize}
                \item Necessary because of \textbf{PID} and \textbf{network} virtualization
                \item Allows addressing MPI processes
            \end{itemize}
        \end{itemize}
    \end{itemize}
}

\frame{\frametitle{Workload}
    NAS Parallel Benchmarks
    \begin{itemize}
        \item Synthetic workload for baseline purposes
        \item EP (\textit{embarrassingly parallel}) Kernel
        \begin{itemize}
            \item CPU bound scenario
            \item Low communication between workers
            \item Workload size B
        \end{itemize}
    \end{itemize}
    \pause
    \vspace*{0.2cm}
    Ondes3D
    \begin{itemize}
        \item Seismic waves propagation simulation
        \item Load imbalance, frequent communication
        \begin{itemize}
            \item Workload: default test case + Ligurian
        \end{itemize}
    \end{itemize}
    \vspace*{0.2cm}
    \pause
    Ping-Pong benchmark
    \begin{itemize}
        \item Network latency
    \end{itemize}
}

\section{Results}
\frame{\tableofcontents[currentsection,hideothersubsections]}

\frame{\frametitle{Results - NAS EP}
    \begin{figure}
        \centering
        \begin{subfigure}{.45\textwidth}
            \hspace*{-1.2cm}
            \includegraphics[width=1.2\textwidth]{nas-ep-raw.png}
        \end{subfigure}
        \pause
        \begin{subfigure}{.45\textwidth}
        \hspace*{0.2cm}
            \includegraphics[width=1.2\textwidth]{nas-ep-overhead.png}
        \end{subfigure}
    \end{figure}
}

\frame{\frametitle{Results - Ondes3D Default}
    \begin{figure}
        \centering
        \begin{subfigure}{.45\textwidth}
            \hspace*{-1.2cm}
            \includegraphics[width=1.2\textwidth]{ondes3d-default-raw.png}
        \end{subfigure}
        \pause
        \begin{subfigure}{.45\textwidth}
        \hspace*{0.2cm}
            \includegraphics[width=1.2\textwidth]{ondes3d-default-overhead.png}
        \end{subfigure}
    \end{figure}
}

\frame{\frametitle{Results - Ondes3D Ligurian}
    \begin{figure}
        \centering
        \begin{subfigure}{.45\textwidth}
            \hspace*{-1.2cm}
            \includegraphics[width=1.2\textwidth]{ondes3d-ligurian-raw.png}
        \end{subfigure}
        \pause
        \begin{subfigure}{.45\textwidth}
        \hspace*{0.2cm}
            \includegraphics[width=1.2\textwidth]{ondes3d-ligurian-overhead.png}
        \end{subfigure}
    \end{figure}
}

\frame{\frametitle{Results - Ping Pong}
    \begin{figure}
        \centering
        \vspace*{-0.2cm}
        \includegraphics[width=0.65\textwidth]{network-latency-raw.png}
    \end{figure}
}

\frame{\frametitle{Results - Alpine Linux}
    \begin{figure}
        \centering
        \begin{subfigure}{.45\textwidth}
            \hspace*{-1.2cm}
            \includegraphics[width=1.2\textwidth]{alpine-nas-ondes3d-raw.png}
        \end{subfigure}
        \pause
        \begin{subfigure}{.45\textwidth}
        \hspace*{0.2cm}
            \includegraphics[width=1.2\textwidth]{alpine-nas-ondes3d-overhead.png}
        \end{subfigure}+
    \end{figure}
}

\section{Conclusions and Future Work}
\frame{\tableofcontents[currentsection,hideothersubsections]}

\frame{\frametitle{Conclusions}
    \begin{itemize}
        \item Containers are a viable way of virtualizing an HPC environment
        \pause
        \item Computational overhead is negligible in most use cases
        \pause
        \item Communication overhead is significant, especially in the Docker environment
        \begin{itemize}
            \item Virtualization of the \textbf{network} namespace impacts MPI workloads
        \end{itemize}
        \pause
        \item Performance gains are attainable by \textbf{fine tuning} the execution environment
    \end{itemize}
}

\frame{\frametitle{Future Work}
    \begin{itemize}
        \item Further investigate Docker Swarm scalability
        \begin{itemize}
            \item Increased latency
            \item Failure to spawn a larger number of containers
        \end{itemize}
        \pause
        \vspace*{0.3cm}
        \item Explore container compatibility with other devices
        \begin{itemize}
            \item GPU
            \item InfiniBand
        \end{itemize}
    \end{itemize}
}

\end{document}
